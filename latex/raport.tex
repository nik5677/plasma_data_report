%! Author = nikodem
%! Date = 24/01/2026

% Preamble
\documentclass[11pt]{article}

% Packages
\usepackage{amsmath}
\usepackage{graphicx}
\usepackage{hyperref}
\usepackage{polski}

% Hyperref configuration
\hypersetup{
    colorlinks=true,
    linkcolor=cyan,
    filecolor=magenta,
    urlcolor=blue
}

% Graphics directory
\graphicspath{ {./figures/} }

% Document
\begin{document}

\section{Wstęp}\label{sec:wstep}

\subsection{Cel pracy}\label{subsec:cel-pracy}

Celem niniejszej pracy jest analiza statystyczna oraz modelowanie dynamiki szeregu czasowego pochodzącego z rzeczywistego eksperymentu fizycznego.
Przedmiotem badań jest natężenie prądu plazmy (ang.\ \emph{Plasma Current}, ) zarejestrowane w urządzeniu typu tokamak.

Głównym zadaniem jest weryfikacja hipotezy o możliwości opisu fluktuacji prądu w fazie stabilnej (tzw.\ \emph{flat-top}) za pomocą liniowego modelu stochastycznego klasy ARMA. Analiza obejmuje zbadanie stacjonarności procesu, identyfikację rzędu modelu, estymację parametrów oraz weryfikację założeń dotyczących reszt.

\subsection{Opis i źródło danych}\label{subsec:opis-i-zrodo-danych}

Dane wykorzystane w projekcie pochodzą z reaktora fuzyjnego MAST (Mega Ampere Spherical Tokamak), znajdującego się w Culham Centre for Fusion Energy w Wielkiej Brytanii.
Zostały pobrane za pośrednictwem otwartego interfejsu API udostępnionego w ramach projektu \href{https://mastapp.site/index.html}{FAIR-MAST}.

Interpretacja fizyczna zmiennej:
Analizowana zmienna to natężenie prądu plazmy \( ( I_p ) \), wyrażone w kiloamperach \( (\text{kA}) \).
W uproszczeniu, parametr ten jest kluczowym wskaźnikiem ,,życia'' eksperymentu:

\begin{itemize}
    \item Wzrost prądu oznacza formowanie się plazmy.
    \item Utrzymywanie stałej wartości (plateau) oznacza fazę stabilną, w której przeprowadza się właściwe eksperymenty.
    \item Nagły spadek wartości do zera może sygnalizować niekontrolowaną utratę stabilności (tzw.\ disruption).
\end{itemize}

Dla potrzeb analizy szeregów czasowych, fluktuacje tego prądu w fazie stabilnej traktujemy jako proces stochastyczny, wynikający z turbulencji wewnątrz gorącego gazu oraz działania systemów sterowania reaktora.

\subsection{Charakterystyka próby i wizualizacja}\label{subsec:charakterystyka-proby-i-wizualizacja}

Do analizy wybrano eksperyment (tzw.\ ,,strzał'') o numerze ID: 30421.

Pełny przebieg eksperymentu trwa niespełna 1/2 sekundy.
Ze względu na niestacjonarny charakter całego procesu (faza rozruchu i wygaszania), do modelowania ARMA wyodrębniono wycinek czasowy odpowiadający fazie stabilnej.

\begin{itemize}
    \item Pełny zbiór: Czas około 6 s.
    \item Zbiór analityczny (wycinek): 0.37 s (1850 obserwacji).
\end{itemize}

Poniższe wykresy prezentują surowe dane pomiarowe (rysunek~\ref{fig:fig1} oraz~\ref{fig:fig2}).

\begin{figure}[h]
    \centering
    \includegraphics[width=1\textwidth]{figure1}
    \caption{Pełny przebieg natężenia prądu plazmy w czasie eksperymentu. Czerwonymi liniami zaznaczono obszar fazy stabilnej wybrany do analizy.}
    \label{fig:fig1}
\end{figure}

\begin{figure}[h]
    \centering
    \includegraphics[width=1\textwidth]{figure2}
    \caption{Wyodrębniony fragment szeregu czasowego (faza \emph{flat-top}) poddany modelowaniu ARMA.}
    \label{fig:fig2}
\end{figure}

\end{document}