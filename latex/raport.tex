%! Author = nikodem
%! Date = 24/01/2026

% Preamble
\documentclass[11pt]{article}

% Packages
\usepackage{amsmath}
\usepackage{graphicx}
\usepackage{hyperref}
\usepackage[all]{hypcap}
\usepackage{polski}
\usepackage{xcolor}
\usepackage[a4paper, total={6in, 8in}]{geometry}
\usepackage{multirow}

\definecolor{darkorchid}{HTML}{9932CC}
% Hyperref configuration
\hypersetup{
    colorlinks=true,
    linkcolor=darkorchid,
    filecolor=magenta,
    urlcolor=blue
}

% Graphics directory
\graphicspath{ {./figures/} }

% Title and authors

\title{\textbf{Analiza danych rzeczywistych natężenia prądu plazmy przy pomocy modelu ARMA}}
\author{Dominika Lewandowska, Nikodem Drelak}

% Document
\begin{document}

    \maketitle
{
  \hypersetup{linkcolor=black} % Tymczasowo zmień kolor na czarny
  \tableofcontents
}


    \section{Wstęp}\label{sec:wstep}

    \subsection{Cel pracy}\label{subsec:cel-pracy}

    Celem niniejszej pracy jest analiza statystyczna oraz modelowanie dynamiki szeregu czasowego pochodzącego z rzeczywistego eksperymentu fizycznego.
    Przedmiotem badań jest natężenie prądu plazmy (ang.\ \emph{Plasma Current}) zarejestrowane w urządzeniu typu tokamak.

    Głównym zadaniem jest weryfikacja hipotezy o możliwości opisu fluktuacji prądu w fazie stabilnej (tzw.\ \emph{flat-top}) za pomocą liniowego modelu stochastycznego klasy ARMA. Analiza obejmuje zbadanie stacjonarności procesu, identyfikację rzędu modelu, estymację parametrów oraz weryfikację założeń dotyczących reszt.

    \subsection{Opis i źródło danych}\label{subsec:opis-i-zrodo-danych}

    Dane wykorzystane w projekcie pochodzą z reaktora fuzyjnego MAST (Mega Ampere Spherical Tokamak), znajdującego się w Culham Centre for Fusion Energy w Wielkiej Brytanii.
    Zostały pobrane za pośrednictwem otwartego interfejsu API udostępnionego w ramach projektu \href{https://mastapp.site/index.html}{FAIR-MAST}.

    Interpretacja fizyczna zmiennej:
    Analizowana zmienna to natężenie prądu plazmy \( ( I_p ) \), wyrażone w kiloamperach \( (\text{kA}) \).
    W uproszczeniu, parametr ten jest kluczowym wskaźnikiem ,,życia'' eksperymentu:

    \begin{itemize}
        \item Wzrost prądu oznacza formowanie się plazmy.
        \item Utrzymywanie stałej wartości (plateau) oznacza fazę stabilną, w której przeprowadza się właściwe eksperymenty.
        \item Nagły spadek wartości do zera może sygnalizować niekontrolowaną utratę stabilności (tzw.\ disruption).
    \end{itemize}

    Dla potrzeb analizy szeregów czasowych, fluktuacje tego prądu w fazie stabilnej traktujemy jako proces stochastyczny, wynikający z turbulencji wewnątrz gorącego gazu oraz działania systemów sterowania reaktora.

    \subsection{Charakterystyka próby i wizualizacja}\label{subsec:charakterystyka-proby-i-wizualizacja}

    Do analizy wybrano eksperyment (tzw.\ \emph{shot}) o numerze ID: 30421.
    Pełny przebieg eksperymentu (przedstawiony na rysunku~\ref{fig:fig1}) trwa około 6 sekund.
    Ze względu na niestacjonarny charakter całego procesu (faza rozruchu i wygaszania), do modelowania ARMA wyodrębniono wycinek czasowy odpowiadający fazie stabilnej (rysunek~\ref{fig:fig2}).
    Zbiór analityczny odpowiada 0,37 sekundy obserwacji i zawiera 1850 próbek.
    W celu późniejszej weryfikacji poprawności modelowania ARMA dla przyszłych wartości, został on podzielony na zbiór treningowy - pierwsze 0,32 sekundy (1500 próbek) oraz zbiór testowy - ostatnie 0.05 sekund (250 próbek).

    \begin{figure}[hp]
        \centering
        \includegraphics[width=1\textwidth]{figure1}
        \caption{Pełny przebieg natężenia prądu plazmy w czasie eksperymentu. Półprzezroczystym obszarem zaznaczono obszar fazy stabilnej wybrany do analizy.}
        \label{fig:fig1}
    \end{figure}

    \begin{figure}[hp]
        \centering
        \includegraphics[width=1\textwidth]{figure2}
        \caption{Wyodrębniony fragment szeregu czasowego (faza \emph{flat-top}) poddany modelowaniu ARMA. Półprzezroczystym obszarem zaznaczono obszar zbioru testowego.}
        \label{fig:fig2}
    \end{figure}


    \section{Przygotowanie danych do analizy}\label{sec:przygotowanie}

    \subsection{Analiza jakości danych}\label{subsec:analiza_jakosci}

    Na wstępnym etapie analizy dokonano oceny jakości danych pomiarowych. Sprawdzono obecność braków danych, wartości odstających oraz nieciągłości czasowych. Analizowany szereg czasowy nie zawiera brakujących obserwacji ani duplikatów, a odstępy czasowe pomiędzy kolejnymi próbkami są jednorodne.\newline
    Wartości natężenia prądu mieszczą się w zakresie fizycznie uzasadnionym dla pracy tokamaka i nie zaobserwowano anomalii mogących wskazywać na błędy pomiarowe. Na tej podstawie dane uznano za poprawne i odpowiednie do dalszej analizy statystycznej.

    \subsubsection{Analiza autokorelacji surowych danych}
    W celu wstępnej identyfikacji struktury zależności czasowych w szeregu obliczono funkcję autokorelacji (ACF) oraz funkcję częściowej autokorelacji (PACF). \newline
    Funkcja autokorelacji ACF opisuje stopień liniowej zależności pomiędzy obserwacjami oddzielonymi o h kroków czasowych i dana jest wzorem:
    \begin{equation}
        \rho(h)=\frac{Cov(X_t,X_{t-h})}{Var(X_t)}.
    \end{equation}
    Z kolei funkcja PACF mierzy korelację pomiędzy $X_t$ i $X_{t-h}$ po wyeliminowaniu wpływu opóźnień pośrednich $1,2,\dots,h-1$, co pozwala na lepszą identyfikację rzędu części autoregresyjnej modelu.
    Rysunek~\ref{fig:empiryczna_acf_surowe} przedstawia wykres empirycznej funkcji autokorelacji ACF dla surowych danych ze zbioru treningowego. Widoczna jest bardzo wolno malejąca autokorelacja oraz istotne wartości współczynnika dla dużych opóźnień, znacząco przekraczające granice przedziału ufności. Taki kształt ACF jest charakterystyczny dla procesów niestacjonarnych.

    \begin{figure}[hp]
        \centering
        \includegraphics[width=1\textwidth]{empiryczna_acf_surowe}
        \caption{Wykres współczynnika autokorelacji surowych danych ze zbioru treningowego w zależności od opóźnienia (parametru $h$). Półprzezroczystym obszarem zaznaczono przedział ufności o poziomie 95\% dla hipotezy o braku korelacji (szum biały).}
        \label{fig:empiryczna_acf_surowe}
    \end{figure}

    Rysunek~\ref{fig:empiryczna_pacf_surowe} przedstawia wykres funkcji częściowej autokorelacji PACF. Również w tym przypadku obserwuje się istotne wartości dla wielu opóźnień, co potwierdza brak stacjonarności oraz obecność silnej struktury trendowej w danych.

    \begin{figure}[hp]
        \centering
        \includegraphics[width=1\textwidth]{empiryczna_pacf_surowe}
        \caption{Wykres współczynnika częściowej autokorelacji surowych danych ze zbioru treningowego w zależności od opóźnienia (parametru $h$). Półprzezroczystym obszarem zaznaczono przedział ufności o poziomie 95\% dla hipotezy o braku korelacji (szum biały).}
        \label{fig:empiryczna_pacf_surowe}
    \end{figure}

    Dodatkowo przeprowadzono test Augmented Dickey–Fullera (ADF), weryfikujący hipotezę zerową o istnieniu pierwiastka jednostkowego. Otrzymano statystykę testową równą −0.665 oraz wartość p równą 0.427, co oznacza brak podstaw do odrzucenia hipotezy o niestacjonarności szeregu.

    \subsection{Dekompozycja szeregu czasowego}\label{subsec:dekompozycja}

    \subsubsection{Różnicowanie danych}
    W celu uzyskania stacjonarności szeregu zastosowano różnicowanie pierwszego rzędu. Operacja ta polega na zastąpieniu oryginalnego szeregu $X_t$ nowym szeregiem:
    \begin{equation}
        Y_t=X_t-X_{t-1},
    \end{equation}
    który opisuje zmiany wartości natężenia prądu pomiędzy kolejnymi chwilami czasu. Różnicowanie jest standardową metodą eliminacji trendu oraz wolnozmiennych składowych deterministycznych w analizie szeregów czasowych. Rysunek~\ref{fig:dekompozycja} przedstawia przebieg szeregu po różnicowaniu. Widoczne jest usunięcie trendu oraz oscylowanie wartości wokół zera, co wskazuje na poprawę własności stacjonarnych. 

    \begin{figure}[hp]
        \centering
        \includegraphics[width=1\textwidth]{dekompozycja}
        \caption{Szereg czasowy ze zbioru treningowego poddany różnicowaniu. Przedstawia zmianę natężenia prądu w zależności od czasu.}
        \label{fig:dekompozycja}
    \end{figure}

    \subsubsection{Ocena autokorelacji danych po dekompozycji}
    Na rysunkach ~\ref{fig:empiryczna_acf_po_dekompozycji}  i ~\ref{fig:empiryczna_pacf_po_dekompozycji} przedstawiono odpowiednio wykresy ACF oraz PACF dla danych po różnicowaniu. W przeciwieństwie do danych surowych, autokorelacje szybko zanikają i w większości mieszczą się w granicach przedziału ufności. Świadczy to o skutecznym usunięciu niestacjonarności.
    \begin{figure}[hp]
        \centering
        \includegraphics[width=1\textwidth]{empiryczna_acf_po_dekompozycji}
        \caption{Wykres współczynnika autokorelacji danych ze zbioru treningowego poddanych różnicowaniu w zależności od opóźnienia (parametru $h$). Półprzezroczystym obszarem zaznaczono przedział ufności o poziomie 95\% dla hipotezy o braku korelacji (szum biały).}
        \label{fig:empiryczna_acf_po_dekompozycji}
    \end{figure}


    \begin{figure}[hp]
        \centering
        \includegraphics[width=1\textwidth]{empiryczna_pacf_po_dekompozycji}
        \caption{Wykres współczynnika częściowej autokorelacji danych ze zbioru treningowego poddanych różnicowaniu w zależności od opóźnienia (parametru $h$). Półprzezroczystym obszarem zaznaczono przedział ufności o poziomie 95\% dla hipotezy o braku korelacji (szum biały).}
        \label{fig:empiryczna_pacf_po_dekompozycji}
    \end{figure}

    Test ADF przeprowadzony dla szeregu po dekompozycji dał statystykę −14.148 oraz wartość p równą 0.0, co jednoznacznie prowadzi do odrzucenia hipotezy o niestacjonarności. Oznacza to, że uzyskany szereg jest stacjonarny i może być modelowany przy użyciu modeli klasy ARMA.


    \section{Modelowanie przy pomocy ARMA}\label{sec:modelowanie_ARMA}

    \subsection{Dobranie rzędu modelu i parametrów}\label{subsec:dobranie_arma}
    Model ARMA$(p,q)$ łączy w sobie składnik autoregresyjny rzędu 
    $p$ oraz składnik średniej ruchomej rzędu $q$. Jego postać ogólna dana jest wzorem:
    \begin{equation}
        X_t=\sum_{i=1}^p \Phi_i X_{t-1} + \sum_{j=1}^q\theta_j\epsilon_{t-1}+\epsilon_t,
    \end{equation}
    gdzie $\epsilon_t$ jest białym szumem. \newline
    Dobór rzędów $p$ i $q$ przeprowadzono na podstawie kryteriów informacyjnych Akaikego ($AIC$) oraz Bayesowskiego ($BIC$), zdefiniowanych jako:
    \begin{equation}
        AIC= -2\ln(L)+2k, \newline
        BIC=-2\ln(L)+k\ln(n),
    \end{equation}
    gdzie $L$ jest wartością funkcji wiarygodności, $k$ liczbą parametrów, a $n$ liczbą obserwacji. \newline
    Na podstawie heatmap przedstawionych na rysunku~\ref{fig:dopasowanie_modelow} stwierdzono, że minimum obu kryteriów osiągane jest dla modelu ARMA(5,5). Parametry modelu zostały wyznaczone metodą największej wiarygodności (MLE), a ich wartości zestawiono w tabeli~\ref{tab:tabela_parametry}.

    \begin{figure}[hp]
        \centering
        \includegraphics[width=1\textwidth]{dopasowanie_modelow}
        \caption{Heatmapa wartości kryteriów AIC oraz BIC w zależności od rzędów $p$ i $q$ modelu ARMA($p, q$) badanego szeregu. Jaśniejszy kolor oznacza niższą wartość kryterium.}
        \label{fig:dopasowanie_modelow}
    \end{figure}
    % Please add the following required packages to your document preamble:
% \usepackage{multirow}
\begin{table}[hp]
\centering
\caption{Parametry modelu ARMA(5, 5) dopasowanego do szeregu ze zbioru treningowego, wyznaczone metodą największej wiarygodności (MLE).\\}
\label{tab:tabela_parametry}
\begin{tabular}{|l|l|l|l|l|l|}
\hline
\multirow{2}{*}{AR(5)} & $\phi_1$    & $\phi_2$   & $\phi_3$   & $\phi_4$    & $\phi_5$   \\ \cline{2-6}
                       & -0.57475292 & 0.61155579 & 0.5790856  & -0.55870469 & 0.97329654 \\ \hline
\multirow{2}{*}{MA(5)} & $\theta_1$  & $\theta_2$ & $\theta_3$ & $\theta_4$  & $\theta_5$ \\ \cline{2-6}
                       & -0.53449985 & 0.5807027  & 0.601362   & -0.55516097 & 0.9431987  \\ \hline
\end{tabular}
\end{table}

    \subsection{Ocena dopasowania modelu}\label{subsec:ocena_arma}

    \subsubsection{Porównanie empirycznych i teoretycznych ACV i PACV}
    Rysunki ~\ref{fig:teoretycznaczna_acf} i ~\ref{fig:teoretycznaczna_pacf} przedstawiają porównanie empirycznych oraz teoretycznych funkcji ACF i PACF dla dopasowanego modelu ARMA(5,5). Widoczna jest bardzo dobra zgodność przebiegów, a większość wartości empirycznych mieści się w granicach przedziałów ufności. Świadczy to o poprawnym odwzorowaniu struktury zależności czasowych przez model.

    \begin{figure}[hp]
        \centering
        \includegraphics[width=1\textwidth]{teoretycznaczna_acf}
        \caption{Wykres empirycznego i teoretycznego współczynnika autokorelacji modelu ARMA(5, 5) dopasowanego do zbioru treningowego, w zależności od opóźnienia (parametru $h$). Półprzezroczystym obszarem zaznaczono przedział ufności o poziomie 95\% wyznaczony metodą Monte Carlo (N=1000).}
        \label{fig:teoretycznaczna_acf}
    \end{figure}

    \begin{figure}[hp]
        \centering
        \includegraphics[width=1\textwidth]{teoretycznaczna_pacf}
        \caption{Wykres empirycznego i teoretycznego współczynnika częściowej autokorelacji modelu ARMA(5, 5) dopasowanego do zbioru treningowego, w zależności od opóźnienia (parametru $h$). Półprzezroczystym obszarem zaznaczono przedział ufności o poziomie 95\% wyznaczony metodą Monte Carlo (N=1000).}
        \label{fig:teoretycznaczna_pacf}
    \end{figure}

    \subsubsection{Analiza trajektorii ARMA w porównaniu do badanego szeregu}
    Rysunek~\ref{fig:przedzialy_ufnosc_ARMA} przedstawia porównanie trajektorii rzeczywistego szeregu po różnicowaniu z trajektorią generowaną przez model ARMA(5,5). Model poprawnie odwzorowuje dynamikę zmian oraz amplitudę fluktuacji.\newline
    Na rysunku~\ref{fig:przedzialy_ufnosc_ARMA_PRZEWIDYWANA} zaprezentowano prognozę modelu dla zbioru testowego. Większość obserwacji rzeczywistych mieści się w 95\% przedziałach ufności, co potwierdza dobrą jakość predykcyjną modelu w krótkim horyzoncie czasowym.

    \begin{figure}[hp]
        \centering
        \includegraphics[width=1\textwidth]{przedzialy_ufnosc_ARMA}
        \caption{Trajektoria szeregu czasowego ze zbioru treningowego poddanego różnicowaniu oraz trajektoria dopasowanego modelu ARMA(5, 5) wraz z przedziałami ufności o poziomie 95\%.}
        \label{fig:przedzialy_ufnosc_ARMA}
    \end{figure}

    \begin{figure}[hp]
        \centering
        \includegraphics[width=1\textwidth]{przedzialy_ufnosc_ARMA_PRZEWIDYWANA}
        \caption{Trajektoria szeregu czasowego ze zbioru treningowego i ze zbioru testowego poddanych różnicowaniu, oraz prognoza trajektorii dopasowanego modelu ARMA(5, 5) dla zbioru testowego wraz z przedziałami ufności o poziomie 95\%.}
        \label{fig:przedzialy_ufnosc_ARMA_PRZEWIDYWANA}
    \end{figure}

    \section{Weryfikacja założeń dotyczących szumu}\label{sec:weryfikacja-zaozen-dotyczacych-szumu}

    \subsection{Założenie dotyczące średniej}\label{subsec:zaozenie-dotyczace-sredniej}
    Przeprowadzono test t-Studenta dla średniej reszt. Otrzymana wartość p = 0.9389 nie daje podstaw do odrzucenia hipotezy zerowej, co oznacza, że średnia reszt jest statystycznie równa zero (rysunek~\ref{fig:test_srednie}).

    \begin{figure}[hp]
        \centering
        \includegraphics[width=1\textwidth]{test_srednie}
        \caption{Wykres wartości reszt modelu ARMA(5, 5) dopasowanego do zbioru treningowego. Czerwoną przerywaną linią zaznaczono średnią reszt o wartości około $-0.00123$.}
        \label{fig:test_srednie}
    \end{figure}


    \subsection{Założenie dotyczące wariancji}\label{subsec:zaozenie-dotyczace-wariancji}
    W celu weryfikacji założenia o stałej wariancji reszt zastosowano test ARCH (Autoregressive Conditional Heteroskedasticity). Test ten sprawdza, czy wariancja reszt w danym momencie czasu zależy od kwadratów reszt z poprzednich chwil, co jest charakterystyczne dla procesów o zmiennej zmienności.

    Hipoteza zerowa testu ARCH zakłada brak efektu ARCH, czyli stałą wariancję reszt w czasie. Odrzucenie hipotezy zerowej oznacza występowanie heteroskedastyczności warunkowej, czyli zmiennej wariancji zależnej od przeszłych zaburzeń losowych.

    Otrzymana wartość p = 0.0000 prowadzi do jednoznacznego odrzucenia hipotezy zerowej, co wskazuje na obecność efektu ARCH w resztach modelu ARMA. Oznacza to, że wariancja procesu nie jest stała w czasie, a model ARMA nie opisuje w pełni struktury zmienności danych.\newline
    W celu lokalizacji zmiany reżimu wariancji wykorzystano sumy skumulowanych kwadratów reszt:
    \begin{equation}
        C_j=\sum_{i=1}^j\epsilon_i^2,
    \end{equation}
    oraz funkcję błedu:
    \begin{equation}
        V(j)=\frac{j}{n}C_n-C_j.
    \end{equation}
    Rysunek~\ref{fig:test_wariancja} wskazuje punkt $l$, w którym zmiana wariancji jest największa.

    \begin{figure}[hp]
        \centering
        \includegraphics[width=1\textwidth]{test_wariancja}
        \caption{Wykresy przedstawiające kolejno: trajektorię reszt modelu ARMA(5, 5) dopasowanego do zbioru treningowego, sumę skumulowanych kwadratów reszt $C_j$ oraz sumę błędów kwadratowych $V(j)$, w zależności od czasu. Czerwoną linią przerywaną zaznaczono punkt $l$, czyli punkt największej zmiany reżimu wariancji.}
        \label{fig:test_wariancja}
    \end{figure}


    \subsection{Założenie dotyczące niezależności}\label{subsec:zaozenie-dotyczace-niezaleznosci}
    Do weryfikacji założenia o niezależności reszt zastosowano test Ljunga–Boxa, który bada, czy grupa autokorelacji do zadanego opóźnienia różni się istotnie od zera. Test ten sprawdza hipotezę zerową, że reszty są nieskorelowane, czyli mają charakter białego szumu.

    Otrzymana wartość p = 0.2137 nie daje podstaw do odrzucenia hipotezy zerowej, co oznacza, że w resztach nie występuje istotna autokorelacja, a model poprawnie opisuje zależności czasowe w średniej procesu.
    \subsubsection{Autokorelacja reszt}
    Reszty (błędy) w modelu ARMA/ARIMA reprezentują różnicę między obserwowanymi wartościami szeregu czasowego a wartościami przewidywanymi (dopasowanymi) przez model.
    Poprawne reszty powinny zachowywać się jak „biały szum”, czyli mieć stałą wariancję, zerową średnią i brak autokorelacji.
    Rysunki ~\ref{fig:empiryczna_acf_po_dekompozycji_RESZTY} i ~\ref{fig:empiryczna_pacf_po_dekompozycji_RESZTY} przedstawiają wykresy ACF i PACF reszt. Wszystkie współczynniki mieszczą się w granicach przedziałów ufności, co potwierdza brak autokorelacji i spełnienie założenia białego szum.

    \begin{figure}[hp]
        \centering
        \includegraphics[width=1\textwidth]{empiryczna_acf_po_dekompozycji_RESZTY}
        \caption{Wykres współczynnika autokorelacji reszt modelu ARMA(5, 5) w zależności od opóźnienia (parametru $h$). Półprzezroczystym obszarem zaznaczono przedział ufności o poziomie 95\% dla hipotezy o braku korelacji (szum biały).}
        \label{fig:empiryczna_acf_po_dekompozycji_RESZTY}
    \end{figure}


    \begin{figure}[hp]
        \centering
        \includegraphics[width=1\textwidth]{empiryczna_pacf_po_dekompozycji_RESZTY}
        \caption{Wykres współczynnika częściowej autokorelacji reszt modelu ARMA(5, 5) w zależności od opóźnienia (parametru $h$). Półprzezroczystym obszarem zaznaczono przedział ufności o poziomie 95\% dla hipotezy o braku korelacji (szum biały).}
        \label{fig:empiryczna_pacf_po_dekompozycji_RESZTY}
    \end{figure}

    \subsection{Założenie dotyczące normalności rozkładu}\label{subsec:zaozenie-dotyczace-normalnosci-rozkadu}
    Normalność rozkładu reszt zweryfikowano przy użyciu testu Jarque–Bera, który opiera się na analizie skośności i kurtozy rozkładu. Hipoteza zerowa testu zakłada zgodność rozkładu reszt z rozkładem normalnym.
    Otrzymana wartość p = 0.0000 prowadzi do odrzucenia hipotezy zerowej, co oznacza, że rozkład reszt istotnie odbiega od normalnego. Tego typu zachowanie jest często obserwowane w danych pochodzących z procesów fizycznych o charakterze turbulentnym. 
    Histogram oraz wykres QQ (Rysunek~\ref{fig:test_normalnosc}) wskazują na cięższe ogony rozkładu, co jest typowe dla danych fizycznych pochodzących z procesów turbulentnych.

    \begin{figure}[hp]
        \centering
        \includegraphics[width=1\textwidth]{test_normalnosc}
        \caption{Histogram i empiryczna gęstość oraz wykres kwantylowy (QQ-plot) wartości reszt w porównaniu do rozkładu normalnego (czerwona, przerywana linia) o parametrach wyznaczonych na podstawie badanej próby reszt (średniej i wariancji z próby).}
        \label{fig:test_normalnosc}
    \end{figure}

    \section{Wnioski}\label{sec:wnioski}
    
    Przeprowadzona analiza natężenia prądu plazmy w fazie stabilnej eksperymentu tokamakowego pozwoliła na ocenę możliwości opisu badanego szeregu czasowego przy użyciu modelu ARMA. Wykazano, że surowe dane są niestacjonarne, jednak zastosowanie różnicowania pierwszego rzędu skutecznie doprowadziło do uzyskania szeregu stacjonarnego, co potwierdziły zarówno analiza ACF i PACF, jak i test Augmented Dickey–Fullera.\newline
    
    Na podstawie kryteriów informacyjnych AIC i BIC dobrano model ARMA(5,5), który dobrze odwzorowuje strukturę zależności czasowych obecnych w danych. Analiza dopasowania oraz prognoz krótkoterminowych wykazała, że model poprawnie opisuje średnią dynamikę procesu, a obserwacje rzeczywiste w większości mieszczą się w wyznaczonych przedziałach ufności.\newline
    
    Jednocześnie weryfikacja założeń modelowych ujawniła istotne ograniczenia. Pomimo spełnienia założeń o zerowej średniej i braku autokorelacji reszt, stwierdzono występowanie zmiennej wariancji oraz odstępstwa od normalności rozkładu. Wskazuje to, że model ARMA nie w pełni opisuje statystyczne własności fluktuacji prądu plazmy.

    \section{Podsumowanie}\label{sec:podsumowanie}
    
    W pracy zastosowano klasyczne metody analizy szeregów czasowych do modelowania rzeczywistych danych pochodzących z eksperymentu fizycznego. Uzyskane wyniki pokazują, że modele ARMA mogą być użyteczne do opisu krótkoterminowej dynamiki badanego procesu, jednak ich zastosowanie jest ograniczone w przypadku danych charakteryzujących się heteroskedastycznością i nienormalnością rozkładu. \newline
    
    W związku z tym dalsze badania powinny obejmować wykorzystanie modeli uwzględniających zmienność wariancji, takich jak ARCH lub GARCH, które lepiej odpowiadają naturze analizowanych danych plazmowych.


\clearpage
\phantomsection
\addcontentsline{toc}{section}{Literatura}
\nocite{*}
\bibliographystyle{plain}
\bibliography{bibliografia}

\end{document}